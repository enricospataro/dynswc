\onehalfspacing

\chapter{Conclusioni e sviluppi futuri}\label{ch:end}

\section{Risultati}
Questa tesi di laurea presenta un nuovo approccio alla visualizzazione dinamica di una word cloud semantica. Sono stati valutati quantitativamente diversi algoritmi di disegno sulla base di varie metriche. Tali algoritmi, gi� esistenti in letteratura, sono stati arricchiti con un metodo force directed in modo da gestire al meglio, se possibile, la dinamicit� della word cloud, tenendo conto della correlazione semantica tra le parole. L'algoritmo Cycle Cover sembra essere il migliore da questo punto di vista: la distanza geometrica tra le parole riflette la loro correlazione semantica, pur mantenendo abbastanza discretamente la mappa mentale dell'utente. 

Il calcolo della similarit� tra cluster di word cloud successive, ha consentito una gestione ottimale delle variazioni dei cluster delle parole.

Inoltre, l'applicazione delle procedure di morphing ha permesso di ottenere, tramite l'utilizzo di animazioni, una visualizzazione dinamica e gradevole della word cloud. 

Il tutto � corredato da un'interfaccia grafica di semplice utilizzo, che permette all'utente di interagire con la word cloud e di modificare a proprio piacimento il layout, portando avanti o indietro la visualizzazione, cos� come un qualsiasi media player.

\section{Sviluppi futuri}
In futuro, si pu� intervenire su diversi aspetti. Innanzitutto, si potrebbe testare tale lavoro da un punto di vista qualitativo, definendo nuove metriche che tengano conto dell'interfaccia grafica e affiancando magari uno user study, in modo da verificare l'efficacia e l'utilit� del nostro lavoro sotto il profilo della quantit� di informazione fornita all'utente e, possibilmente, di migliorare il sistema realizzato.

Per quanto riguarda gli aspetti algoritmici, una parte fondamentale � costituita dal preprocessing del testo: l'utilizzo di tecniche pi� sofisticate di text mining (e.g. \textit{Part-of-Speech Tagging}, cio� l'assegnazione di categorie linguistiche alle parole), potrebbe meglio classificare le parole e catturare in modo pi� preciso le loro relazioni semantiche. Un'altra modifica da introdurre, � quella di considerare, tra parole in comune di word cloud consecutive, parole aventi la stessa radice. In tal caso, per�, � necessario intervenire anche a livello di interfaccia grafica, poich� con la procedura di morphing, la trasformazione di una parola in un'altra dovrebbe avvenire in modo graduale.

Anche l'interfaccia grafica � da migliorare. Ad esempio, potrebbe essere interessante sviluppare un'applicazione web, arricchendo il layout con statistiche riguardanti le parole e i cluster.