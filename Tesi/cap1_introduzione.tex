\onehalfspacing

%%CAPITOLO 1: Introduzione=======================================
\chapter{Introduzione}
Il recente sviluppo di Internet, con l'avvento del Web 2.0, assieme al grande progresso tecnologico dei calcolatori, ha comportato un'ingente produzione di dati sul web e sulle piattaforme web-based, per cui il problema di estrarre, gestire e visualizzare efficacemente tale informazione � diventato, negli ultimi anni, un ambito di ricerca piuttosto importante nella visualizzazione dell'informazione.

Una modalit� di visualizzazione dei dati che negli ultimi tempi ha suscitato molto interesse � costituita dalla word cloud (letteralmente \textit{nuvola di parole}), che � una rappresentazione visuale delle parole pi� significative di un documento. In generale, l'utilizzo delle word cloud consente agli utenti di avere una rapida sintesi sul contenuto di un testo, facilitandone la comprensione. I contesti applicativi sono molteplici, e verranno discussi nei capitoli successivi.

L'obiettivo del seguente lavoro di tesi consiste nella realizzazione di un sistema che consente di visualizzare, in modo dinamico, l'evoluzione temporale del contenuto di un testo, tramite l'utilizzo di word cloud. Nello specifico, il testo viene elaborato ad intervalli regolari, da cui si estraggono le parole pi� rilevanti del testo (fino a quell'istante), per poi visualizzarle sullo schermo, tenendo conto delle relazioni semantiche tra le parole. Il tutto avviene in un contesto dinamico, cio� il passaggio da una word cloud alla successiva avviene in modo graduale tramite delle animazioni. 

Rispetto allo stato dell'arte, questo lavoro di tesi si propone di creare word cloud dinamiche relative a brevi periodi temporali (e.g. un discorso), per cui diventa importante avere delle animazioni che mostrino l'evoluzione della word cloud.
\\ \\
La tesi di laurea � dunque strutturata come segue:
\begin{itemize}
\item Nel capitolo \ref{ch:wc_stat} verr� introdotto il concetto di word cloud, facendo il punto sulle caratteristiche, sullo stato dell'arte e sulle possibili applicazioni.
\item Il capitolo \ref{ch:wcdin}, dopo una breve distinzione tra word cloud statiche e dinamiche, tratter� gli aspetti algoritmici del sistema sviluppato. Verranno descritte le varie fasi, composte da diversi algoritmi, che consentono la creazione di una word cloud statica (sezione \ref{wc_din:algs}). L'approccio utilizzato per conferire dinamicit� al sistema � invece esposto nella sezione \ref{wc_din:din_algs}.
\item I dettagli implementativi, relativi all'architettura del sistema, insieme ai risultati dell'analisi sperimentale, fanno parte del capitolo \ref{ch:impl_test}.
\item Infine, il capitolo \ref{ch:end} conclude il lavoro di tesi, riportando le considerazioni finali e delineando possibili sviluppi futuri.
\end{itemize}